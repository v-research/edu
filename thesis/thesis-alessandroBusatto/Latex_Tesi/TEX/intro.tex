\section{Introduzione}

Negli ultimi anni abbiamo assistito all'evoluzione del mondo digitale, la nascita di Internet e lo sviluppo di nuove tecnologie ha cambiato radicalmente le nostre vite, ad esempio le e-mail ci consentono di spedire lettere senza dover andare fisicamente ad imbucarle nella cassetta della posta, è possibile gestire i conti bancari comodamente da casa accedendo all'area personale del sito della banca, controllare gli elettrodomestici da remoto quando non si è a casa, frequentare le lezioni o lavorare da casa quando impossibilitati ad essere presenti in aula o sul luogo di lavoro.\\
Naturalmente tutta quest'evoluzione ci ha portato numerosi vantaggi, ma allo stesso tempo sono emersi nuovi tipi di problematiche legati soprattutto alla gestione e alla protezione dei dati e delle informazioni che viaggiano attraverso la rete.\\
Molti ricercatori hanno cercato di trovare delle soluzioni a queste problematiche dando vita a protocolli di sicurezza che utilizzano la crittografia per mantenere al sicuro la comunicazione.\\
La storia ci insegna che anche protocolli di sicurezza ritenuti sicuri per anni, si sono dimostrati vulnerabili a qualche tipologia di attacco, come ad esempio il protocollo a chiave pubblica descritto da Needham e Schroeder in \cite{NS78} che è risultato vulnerabile ad un attacco descritto da Lowe in \cite{L95}.\\
Per questo motivo si cercano sempre dei nuovi protocolli di sicurezza, o varianti di quelli già esistenti, in grado di mantenere al sicuro i nostri dati.\\
Come si può affermare che un protocollo di sicurezza rispetti determinate specifiche di sicurezza?\\
L'obiettivo di questo documento è analizzare lo stato dell'arte sulle varie tecniche per la verifica dei protocolli e cercare di avvicinare la verifica di protocolli all’ingegneria di sistemi utilizzando la modellazione UML.\\ 
Verrà proposta una nuova tecnica di modellazione per i protocolli di sicurezza mediante l'utilizzo dei diagrammi UML, questo perch\'e le tecniche attualmente esistenti richiedono la conoscenza di un linguaggio di dominio specifico per la modellazione di protocolli da utilizzare nei tool per la verifica automatica.\\
Vedremo che attualmente esistono due tecniche per la verifica dei protocolli, il modello computazionale e il modello simbolico, ma solo il modello simbolico viene considerato abbastanza ``maturo'' per essere utilizzato e si presta ad essere automatizzato, per questo motivo al giorno d'oggi viene utilizzato dai principali tool di verifica automatica.\\
L'utilizzo della modellazione tramite diagrammi UML consente ai progettisti di protocolli di utilizzare una rappresentazione semi-grafica\footnote{un modello UML è composto da un insieme di diagrammi correlati tra loro ed ogni diagramma è costituito da diversi elementi grafici oltre che da elementi di testo libero}, di ragionare sulla struttura del protocollo senza curarsi dell'effettiva implementazione e di utilizzare il tool da me sviluppato, presentato successivamente, per convertire il diagramma UML nel linguaggio adatto per essere utilizzato dal tool di verifica VerifPal\footnote{il codice presentato nella tesi è consultabile presso il sito: \url{edu.v-research.it}}.\\
